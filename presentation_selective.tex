\documentclass[9pt,professionalfont,t,aspectratio=169]{beamer}
\usepackage{amsmath}
\usepackage{amssymb}
\usepackage{amsthm}
\usepackage{empheq}
\usepackage{xparse}
\usepackage[export]{adjustbox}
\usepackage{varwidth}
\usepackage{tabulary}
\usepackage{xcolor}
\usepackage{hhline}
\usepackage{relsize}
\usepackage{framed}
\usepackage[beamer]{hf-tikz}
\usepackage{multirow}

\usepackage[style=authoryear]{biblatex}
\usepackage{bm}
\usepackage{changepage}
\usepackage[T1]{fontenc}
\usepackage{concmath}
\usepackage{mathrsfs}
\usepackage{mathtools}
\usepackage[thinlines]{easytable}
\usepackage{xy}
\usepackage{wrapfig}
\usepackage{pgfplots,tikz}
\usepackage{soul}
\usepackage{verbatim}
\usetikzlibrary{matrix,positioning,arrows}
\usetikzlibrary{tikzmark}
\usetikzlibrary{fit}
\usetheme[progressbar=frametitle]{metropolis}
\usepackage{appendixnumberbeamer}
\usepackage{ wasysym }

\usepackage{booktabs}
%\usepackage[scale=2]{ccicons}
\usepackage{lmodern}
\usepackage{xspace}
\usepackage{graphicx}
\usepackage{mathtools}
\usepackage{float}
\usepackage{amstext}
\usepackage{array}
\usepackage{lipsum}
\usepackage{amsfonts}              
\usepackage{amsthm}               
\usepackage{amssymb}
%\usepackage{multimedia}
\usepackage{media9}
\usepackage[absolute,overlay]{textpos}
\usepackage[font=small]{caption}
\captionsetup[figure]{labelformat=empty}%
\captionsetup[table]{labelformat=empty}%
\usepackage{pgfplots,tikz}
\usepackage{soul}
\usepackage{verbatim}
\usetikzlibrary{matrix,positioning,arrows}
\usetikzlibrary{tikzmark}
\usetikzlibrary{fit}


\DeclareMathOperator{\Avg}{\mathbb{E}}

\newcommand*\numcircledmod[1]{\raisebox{.5pt}{\textcircled{\raisebox{-.9pt} {#1}}}}

%\usepackage{animate}
\usepackage{etoolbox}% http://ctan.org/pkg/etoolbox

\setbeamercolor{block title}{fg=Medalist,bg=white} % Colors of the block titles

\setbeamercolor{block body}{fg=black,bg=white} % Colors of the body of blocks

\setbeamercolor{block alerted title}{fg=white,bg=BerkBlue!80} % Colors of the highlighted block titles

\setbeamercolor{block alerted body}{fg=black,bg=BerkBlue!10} % Colors of the body of highlighted blocks

\setbeamercolor{block title example}{fg=BerkBlue,bg=white} % Colors of the block titles

\setbeamercolor{block body example}{fg=black,bg=white} % Colors of the body of blocks

\setbeamertemplate{caption}{\raggedright\insertcaption\par}

\newcommand\setItemnumber[1]{\setcounter{enumi}{\numexpr#1-1\relax}}

\newcommand{\ch}[1]{{\color{purple}#1}}
\newcommand{\km}[1]{{\color{red}#1}}
\newcommand{\ed}[1]{{\color{blue}#1}}

\newbox\FBox
\NewDocumentCommand\Highlight{O{black}O{white}mO{0.5pt}O{0pt}O{0pt}}{%
    \setlength\fboxsep{#4}\sbox\FBox{\fcolorbox{#1}{#2}{#3\rule[-#5]{0pt}{#6}}}\usebox\FBox}
      
\DeclareCiteCommand{\longcite}{}{%
    \printnames{author}, \printfield{journaltitle} \printfield{year}}{;}{}%
    

\newcommand*\diff{\mathop{}\!\mathrm{d}}
\newcommand*\Diff[1]{\mathop{}\!\mathrm{d^#1}}
\def\*#1{\mathbf{#1}}
\DeclareMathOperator{\Tr}{Tr}

% Syntax: \colorboxed[<color model>]{<color specification>}{<math formula>}
\newcommand*{\colorboxed}{}
\def\colorboxed#1#{%
  \colorboxedAux{#1}%
}
\newcommand*{\colorboxedAux}[3]{%
  % #1: optional argument for color model
  % #2: color specification
  % #3: formula
  \begingroup
    \colorlet{cb@saved}{.}%
    \color#1{#2}%
    \boxed{%
      \color{cb@saved}%
      #3%
    }%
  \endgroup
}



\title{Excitations, Emergent Facilitation and Glassy Dynamics in Supercooled Liquids}
% \date{Virtual Seminar --  Theory Club, \today}
\date{APS March Meeting 2024, March 4, 2024}
\author{Muhammad R. Hasyim\inst{1}, Kranthi K. Mandadapu\inst{2}}
\institute[Affiliations]{
    \inst{1}%
    Department of Chemical and Biomolecular Engineering, University of California, Berkeley, CA 94720
    
    \inst{2}%
    Chemical Division, Lawrence Berkeley National Laboratory, Berkeley, CA 94720
%    \\ 
%    Contact: \url{mh7373@nyu.edu}
}
% \newcommand\FrameText[1]{%
% \begin{textblock*}{\paperwidth}(\textwidth-\widthof{#1},0.92\textheight)\raggedright #1\hspace{0.5em}
% \end{textblock*}}
\metroset{block=fill}
\begin{document}
\documentclass[9pt,professionalfont,t,aspectratio=169]{beamer}
\usepackage{amsmath}
\usepackage{amssymb}
\usepackage{amsthm}
\usepackage{empheq}
\usepackage{xparse}
\usepackage[export]{adjustbox}
\usepackage{varwidth}
\usepackage{tabulary}
\usepackage{xcolor}
\usepackage{hhline}
\usepackage{relsize}
\usepackage{framed}
\usepackage[beamer]{hf-tikz}
\usepackage{multirow}

\usepackage[style=authoryear]{biblatex}
\usepackage{bm}
\usepackage{changepage}
\usepackage[T1]{fontenc}
\usepackage{concmath}
\usepackage{mathrsfs}
\usepackage{mathtools}
\usepackage[thinlines]{easytable}
\usepackage{xy}
\usepackage{wrapfig}
\usepackage{pgfplots,tikz}
\usepackage{soul}
\usepackage{verbatim}
\usetikzlibrary{matrix,positioning,arrows}
\usetikzlibrary{tikzmark}
\usetikzlibrary{fit}
\usetheme[progressbar=frametitle]{metropolis}
\usepackage{appendixnumberbeamer}
\usepackage{ wasysym }

\usepackage{booktabs}
%\usepackage[scale=2]{ccicons}
\usepackage{lmodern}
\usepackage{xspace}
\usepackage{graphicx}
\usepackage{mathtools}
\usepackage{float}
\usepackage{amstext}
\usepackage{array}
\usepackage{lipsum}
\usepackage{amsfonts}              
\usepackage{amsthm}               
\usepackage{amssymb}
%\usepackage{multimedia}
\usepackage{media9}
\usepackage[absolute,overlay]{textpos}
\usepackage[font=small]{caption}
\captionsetup[figure]{labelformat=empty}%
\captionsetup[table]{labelformat=empty}%
\usepackage{pgfplots,tikz}
\usepackage{soul}
\usepackage{verbatim}
\usetikzlibrary{matrix,positioning,arrows}
\usetikzlibrary{tikzmark}
\usetikzlibrary{fit}


\DeclareMathOperator{\Avg}{\mathbb{E}}

\newcommand*\numcircledmod[1]{\raisebox{.5pt}{\textcircled{\raisebox{-.9pt} {#1}}}}

%\usepackage{animate}
\usepackage{etoolbox}% http://ctan.org/pkg/etoolbox

\setbeamercolor{block title}{fg=Medalist,bg=white} % Colors of the block titles

\setbeamercolor{block body}{fg=black,bg=white} % Colors of the body of blocks

\setbeamercolor{block alerted title}{fg=white,bg=BerkBlue!80} % Colors of the highlighted block titles

\setbeamercolor{block alerted body}{fg=black,bg=BerkBlue!10} % Colors of the body of highlighted blocks

\setbeamercolor{block title example}{fg=BerkBlue,bg=white} % Colors of the block titles

\setbeamercolor{block body example}{fg=black,bg=white} % Colors of the body of blocks

\setbeamertemplate{caption}{\raggedright\insertcaption\par}

\newcommand\setItemnumber[1]{\setcounter{enumi}{\numexpr#1-1\relax}}

\newcommand{\ch}[1]{{\color{purple}#1}}
\newcommand{\km}[1]{{\color{red}#1}}
\newcommand{\ed}[1]{{\color{blue}#1}}

\newbox\FBox
\NewDocumentCommand\Highlight{O{black}O{white}mO{0.5pt}O{0pt}O{0pt}}{%
    \setlength\fboxsep{#4}\sbox\FBox{\fcolorbox{#1}{#2}{#3\rule[-#5]{0pt}{#6}}}\usebox\FBox}
      
\DeclareCiteCommand{\longcite}{}{%
    \printnames{author}, \printfield{journaltitle} \printfield{year}}{;}{}%
    

\newcommand*\diff{\mathop{}\!\mathrm{d}}
\newcommand*\Diff[1]{\mathop{}\!\mathrm{d^#1}}
\def\*#1{\mathbf{#1}}
\DeclareMathOperator{\Tr}{Tr}

% Syntax: \colorboxed[<color model>]{<color specification>}{<math formula>}
\newcommand*{\colorboxed}{}
\def\colorboxed#1#{%
  \colorboxedAux{#1}%
}
\newcommand*{\colorboxedAux}[3]{%
  % #1: optional argument for color model
  % #2: color specification
  % #3: formula
  \begingroup
    \colorlet{cb@saved}{.}%
    \color#1{#2}%
    \boxed{%
      \color{cb@saved}%
      #3%
    }%
  \endgroup
}



\title{Excitations, Emergent Facilitation and Glassy Dynamics in Supercooled Liquids}
% \date{Virtual Seminar --  Theory Club, \today}
\date{APS March Meeting 2024, March 4, 2024}
\author{Muhammad R. Hasyim\inst{1}, Kranthi K. Mandadapu\inst{2}}
\institute[Affiliations]{
    \inst{1}%
    Department of Chemical and Biomolecular Engineering, University of California, Berkeley, CA 94720
    
    \inst{2}%
    Chemical Division, Lawrence Berkeley National Laboratory, Berkeley, CA 94720
%    \\ 
%    Contact: \url{mh7373@nyu.edu}
}
% \newcommand\FrameText[1]{%
% \begin{textblock*}{\paperwidth}(\textwidth-\widthof{#1},0.92\textheight)\raggedright #1\hspace{0.5em}
% \end{textblock*}}
\metroset{block=fill}
\begin{document}

\maketitle 


