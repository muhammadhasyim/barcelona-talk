\begin{frame}\label{intermezzo}
\frametitle{Elastic Dynamical Facilitation (EDF) Theory: A Comprehensive Framework}

\begin{block}{\centering Limitations of Traditional DF Theory}
\centering While DF theory explains facilitation phenomenologically, it lacks:
\begin{itemize}
\item<2-> \textbf{Microscopic origin} of excitations and their energy scale
\item<3-> \textbf{Physical mechanism} for onset temperature $T_\mathrm{o}$  
\item<4-> \textbf{Quantitative connection} between structure and dynamics
\end{itemize}
\end{block}

\onslide<5->{
\begin{block}{\centering EDF Theory: Elasticity as the Unifying Principle}
\centering \textbf{Three interconnected pillars} provide a complete microscopic picture:
\end{block}
}

\vspace{-1.0em}
\hspace{-1.5em}\begin{tabulary}{1.05\linewidth}{CCC}
\onslide<6->{\textbf{\Large Excitations}} &  \onslide<7->{\textbf{\Large Onset Temperature}} & \onslide<8->{\textbf{\Large Facilitation}} \\  
\onslide<6->{
%\includegraphics[height=0.25\textwidth]{6.a-intro_beyonddf/Checkmark_exc.pdf}
} & \onslide<7->{
%\includegraphics[height=0.25\textwidth]{6.a-intro_beyonddf/Checkmark_onset.pdf}
} & \onslide<8->{
%\includegraphics[height=0.25\textwidth]{6.a-intro_beyonddf/Checkmarkfacilitation.pdf}
}\\
\vspace{-1em}
\onslide<6->{
{\small \textbf{Pure shear excitations} with energy scale $J_\sigma \sim G R_\mathrm{exc}^2 \epsilon_\mathrm{c}^2$}
} & 
\vspace{-1em}
\onslide<7->{
{\small \textbf{2D melting theory} explains $T_\mathrm{o}$ from elastic stability}
}
& 
\vspace{-1em}
\onslide<8->{{\small \textbf{Elastic stress fields} create emergent facilitation}} \\
\end{tabulary}

\onslide<9->{
\begin{block}{\centering Key Insight}
\centering \textbf{Elasticity} provides the microscopic foundation for all three pillars, creating a \textbf{unified quantitative theory} of glassy dynamics.
\end{block}
}

\end{frame}


